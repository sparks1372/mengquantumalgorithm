\chapter{Glossary}
\noindent\textbf{Quantum Artefact} is the general term used for a solution to a problem using quantum computation.
Quantum algorithms and quantum circuits are both types of quantum artefacts.

\noindent\textbf{Quantum Circuit} is an ordered sequence of quantum gate that affects a fixed number of qubits.

\noindent\textbf{Quantum Algorithm} is an ordered sequence of instructions that when followed create a quantum circuit.
The instructions can use variables such as the system size resulting in different circuits for different values.

\noindent\textbf{Quantum Gate} is a physical implementation that performs a specific unitary operation.

\noindent\textbf{Unitary Operation} is an operation that when its defining matrix is multiplied by its complex conjugate the result is the identity.

\noindent\textbf{System Size} is the number of qubits that a quantum circuit affects.

\noindent\textbf{Ket} represents a basis function of the respective Hilbert Space, $\mathcal{H}$, as a column vector.

\noindent\textbf{Bra} denote the 'dual vector' of the corresponding Ket.

% \noindent\textbf{Hilbert Space} extend the simple Euclidean vector space into a potentially infinite dimension function space.

\noindent\textbf{Hermitian Conjugate} is the matrix produced when each element in the matrix U is replaced by its complex conjugate and then the resulting matrix is transposed.