\chapter{Introduction}
 \setcounter{page}{0} 
\pagenumbering{arabic}

In 1980, Richard Feynman noted `it is impossible to represent the results of quantum mechanics with a classical universal device`\cite{Feynman82simulatingphysics}.
This statement was a seed for interest in the field of Quantum Computation which uses properties of quantum mechanics to perform computation.
The true power of quantum computation was not initially realised.
The discovery of a quantum artefact by David Deutsch\cite{Deutsch85quantumtheory} in 1985 that performed better than a classical computer was the first glimpse of the potential power provided by harnessing quantum mechanics.
The algorithm was able to distinguish between balanced and constant binary functions.
The algorithm only worked for simple functions with inputs limited to $0$ and $1$ but the advantage over its classical equivalent is the number of times the function needs to be evaluated.
The classical equivalent requires two evaluations, once for both possible inputs, yet the quantum algorithm proposed by Deutsch performs a single evaluation.
Deutch's algorithm is explained in detail in Section \ref{sec:DeutAlg}.

However, with slow progress of research into both their implementation and algorithms, the energy behind the research started to decrease.
It would take a discovery by Peter Shor\cite{Shor:1994jg} to reignite the excitement surrounding the subject.
Shor's discovery was not only important because it was one of very few known quantum algorithms but it also challenged one of the foundations of many encryption techniques currently used.
Many cryptographic techniques as based on the belief that factorisation of a large number into its constituent primes would take such a long time that the encrypted information would be useless by the time the factorisation was complete.
Shor's algorithm challenged that belief and provided a polynomial solution to the factorisation problem using quantum computation.
Shor's algorithm is explained in detail in Section \ref{sec:shorsalg}.

\section{Report Structure}
This report has the following structure:
\begin{itemize}
 \item Literature Review - Section \ref{sec:introtoquantcomp}
 \item Requirements - Section \ref{sec:reqs}
 \item Design and Implementation - Section \ref{sec:desandimp}
 \item Testing - Section \ref{sec:testing}
 \item Experimentation - Section \ref{sec:experimentation}
 \item Evaluation and Future Work - Section \ref{sec:evalandfutwork}
\end{itemize}




