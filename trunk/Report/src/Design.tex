\chapter{Design and Implementation}
\lstset{language=Java}
This section shall include both design decisions and implmentation decisions.
Java was selected as the programming language for the implementation.
The main reason for this choice is that Java is widely adopted by many researchers.
The level of tool support is significantly higher than many other programming languages providing better development environments and support tools, debuggers and profilers for example. 
As a result of the requirement for portability, Java provides much simpler support for portability without numerous versions required for each operating system.

Also considered was the availability of libraries likely to be used by researchers.
The research inspiring this framework are all evolutionary search techniques, as a result the availability of the ECJ\cite{ecjtool} strengthed the decision to implement the framework in Java.
As part of the development, a search engine based loosely on the QPACE IV search engine introduced by Massey\cite{masseythesis}.
This search engine implementation is done using ECJ.

\section{Framework}
In this section I will outline the design decisions that only directly effect the framework produced and how it was implemented.

\subsection{Complex Numbers}
Complex numbers are central to Quantum Computing.
As such, any attempt to simulate the behaviour of a Quantum Circuit must handle complex numbers.

There are really only two ways to handle the existence of complex numbers.
One can represent a complex number explicitly as a pair of floating point numbers, or to encapsulate the representation inside a ``complex number'' data structure.

The framework uses the second of these options and provides the ``Complex'' class.
This was chosen for several reasons.
The primary reason was to reduce the risk of programming errors effecting the simulation.
If complex multiplication, addition and other operations had to be replicated throughout the frameworks codebase, and the codebase of any research work, the likelihood of implementation error is much higher, and the tests required to find the error become more specific.
It is much better software engineering practice to encapsulate the properties, real and imaginary values, and the operations on those properties, arithmetic etc.

A season reason is that after brief research online, there are complex number libraries already available.
This reuse of previously written software can also reduce the likelihood of errors in the code.
This is not necessarily due to the software being written by people that are more intelligent or that are better programmers, or even that the software has been explicitly tested more thoroughly than if I were to write a complex number class.
It is due to the size of the deployment footprint.
The number of times the software has previously been deployed, and therefore the number of times it has been implicitly tested by users.

The third reason is that one of the principles behind producing the framework is the attempt to try standardise the research from different researchers.
Without the provision of this ``Complex'' class one researcher could use Cartesian representation, two floating point values, while a second researcher could use the Polar representation, also two floating point values.
If the documentation of the software produced by the two researchers did not mention the representation used, a third researcher could try combine, or compare, the two pieces of software using the framework.
The third researcher is likely to receive very confusing and highly misleading results.
The provision of a Complex class that is used throughout the framework where complex numbers need to be used will reduce the risk of such an event.

The implementation of the Complex class is based on an implementation provided as part of a ``Complex Function Grapher''\cite{compimp}.
The Complex class provides both Cartesian form $real + imaginary$, through \lstinline{real()} and \lstinline{imag()} returning the real and imaginary components respectively, and Polar form $re^{i\theta}$, through \lstinline{mod()} and \lstinline{arg()} returning $r$ and $\theta$ respectively.
The implementation has been adapted to better suit this application.
A calculation of the euclidean distance between two complex numbers is provided.
A second addition is the ability to parseComplex in a similar way to parseInt or parseDouble provided by the standard Integer and Double classes respectively.
This allows a string such as ``$2+3i$'' to be input by a user and for a Complex object to be created with a real component of $2$ and imaginary component of $3$.

\subsection{Matrices}
As seen in Equation \ref{eq:notexplanded} the application of a quantum gate is simply the application of a unitary operation, represented as a matrix, to a quantum state.
This adds the requirement on the framework to provide a manner in which matrices will be represented.

In a similar way to the complex numbers discussed above, there are two distinct ways the framework could have been designed.
The framework could either use an explicit representation, two-dimensional arrays, or could provide a Matrix data structure.

The framework has been designed to use the data structure encapsulation as the matrix representation.
The justification is identical to that discussed above.
Matrix operations are easy to get wrong in implementation and there are matrix libraries for many languages.
The incomplete documentation argument also holds with matrices.
If the framework were to just simply represent matrices as two dimensional arrays, two researchers could order the dimensions differently leading to similar problems to that of conflicting complex representations for the third researcher.

The implementation of the Matrix used in the application is based on the JAMA Matrix package\cite{javamatrix}.
The JAMA Matrix package provides matrices of double values.
For use in the framework this needed to be updated so that it provides matrices of Complex objects and performs all the required operations as though they are matricies of complex numbers.

The JAMA Matrix package provides extra functionality for matricies of doubles that are not required for the framework.
In the conversion from double matrices to Complex matrices this additional functionality was removed.
This was due to the functionality not being required and therefore can be seen as dead code so should be removed.

One matrix operation that was not included in the JAMA Matrix package was the tensor product operation.
This operation is used heavily when applying a single qubit gates to multi-qubit systems and therefore necessary for the framework.

As is explained in Section \ref{sec:testsuitestruc} custom unitary matrices are needed and used as part of the test cases.
This requires these unitary matrices to be stored in a form that can be distributed between researchers.
As with all elements of this framework that needs to be distrubutable, XML is used to store these matrices.
The XML structure can be seen in Figure \ref{}.
The use of XML was used rather than the Java Serializable interface to ensure that 3rd party applications can be used to modify the stored matrices if required.

\subsection{State}
With a representation of matrices defined, the definition of a quantum state naturally followed.
Using the matrix representation, quantum states are defined simply as $2^n\times1$ matrices, vectors.
This representation makes unitary application much simpler as it automatically supported as matrix multiplication.

\subsection{Test Suite Structures}
\label{sec:testsuitestruc}
With most problems there are a series of expected results that are used to measure the suitability of any suggested solution.
The expected results are also usually coupled with the respective inputs.

For Quantum Algorithms the expected results are the state vectors produced by circuits constructed by the algorithm.
As such it was chosen that a test case would be represented as a pair of state vectors, the stating state and the expected state.
The application of a quantum gate is a simple mapping from a starting state to a resulting state.
When a circuit can be defined as a single unitary operation, a custom quantum gate, this representation seems a natural choice.

For some problems, such as the Deutsch and Deutsch-Jozsa problems, the starting and final state are not information but pure data, they have no context.
As is shown in Section \ref{sec:quantumgates} there is no gate $f$ as is used in the Deutsch and Deutsch-Jozsa algorithms.
This may initially seem a rather strange ommision.
However, the $f$ in the Deutsch and Deutsch-Jozsa algorithms are not fixed gates, they are arbitary unitary operations that are guaranteed to be either constant or balanced.
Therefore it is not sufficient for the test cases to be just the starting and final state but must provide a way for custom unitary matrices to be specified.
With these custom matrices specifed, the starting and final states in the test cases for the Deutsch and Deutsch-Jozsa problems are given context and therefore are transformed into meaningful information.

Each circuit produced by the Quantum Algorithms has $n$ qubits.
This means that it can only be evaluated using test cases for $n$ qubits.
Test cases for any other number of qubits would not produce useful results, and are potentially incomputatable due to incorrect matrix dimensions for multiplication.
The notion of a test set was introduced to hold all test cases for a specific $n$.
All test cases are held within a test set.

However, the power of a quantum algorithm over a quantum circuit is the generality of the algorithm for any $n$.
This means that a single test set is not suitable as it would only evaluate the algorithm for test cases for a single $n$.
The notion of a test suite is introduced.
A test suite is used to hold all the test sets produced for the same problem.
There is only one test set for each distinct value of $n$.
The implemented structure can be seen in Figure \ref{fig:testsuiteclassdiag}.

The number of custom gates that are available to use are constant for all the test cases in the test suite.
This number cannot vary as the algorithms produced must be able to contain only the gates available.
If each test case were able to have a different number of custom gates an algorithm produced could contain the instruction to include ``Custom Gate 4'' but a test case could only provide two custom gates.

The test suite is fully defined in a single XML file.
The XML in Figure \ref{code:paulixtestset} is a sample of such a file.
It is easy to see file structure reflects the internal structure of test suites just described.

\lstset{language = XML}
\begin{figure}
 \begin{lstlisting}
 <testsuite>
    <testset NumQubits="1">
      <testcase><!--0-->
	<starting_state>
	  <matrix_element><!-- 0-->
	    <Real>1.0</Real>
	    <Imag>0.0</Imag>
	  </matrix_element>
	  <matrix_element><!-- 1-->
	    <Real>0.0</Real>
	    <Imag>0.0</Imag>
	  </matrix_element>
	</starting_state>
	<final_state>
	  <matrix_element><!-- 0-->
	    <Real>0.0</Real>
	    <Imag>0.0</Imag>
	  </matrix_element>
	  <matrix_element><!-- 1-->
	    <Real>1.0</Real>
	    <Imag>0.0</Imag>
	  </matrix_element>
	</final_state>
      </testcase>
    </testset>
</testsuite>
 \end{lstlisting}
\caption{Partial Test Set for Pauli X Gate}
\label{code:paulixtestset}
\end{figure}

The use of a seperate files to specify the test suite for each Search Problem rather than including it in the Search Problem Manager configuration file, discussed in Section \ref{sec:manclasses}, ensures that the files remain readable.
It also ensure that researchers can distribute Test Suites easier than if all Search Problems were fully specified in a single file.

\begin{figure}
\centering
\begin{emp}[classdiag](20, 20)
Class.A("TestCase")("-StartState:Matrix","-FinalState:Matrix","-Label:String","-Id:int","-CustomMatrices:String[]")();
Class.B("TestSet")("-NumberOfQubits:int","-NumberOfCustomGates:int")();
Class.C("TestSuite")("-NumberOfCustomGates:int")();
% Class.D("SuperPositionalTestSet")()();
leftToRight(75)(C, B, A);
% D.n = B.s - (0, 50);
drawObjects(C, B, A);
clink(association)(C, B);
item(iAssoc)("1")(obj.nw = C.e);
item(iAssoc)("1..*")(obj.ne = B.w);
clink(association)(B, A);
item(iAssoc)("1")(obj.nw = B.e);
item(iAssoc)("1..*")(obj.ne = A.w);
% link(inheritance)(D.n -- B.s)
\end{emp}
\caption{Test Suite Class Diagram}
\label{fig:testsuiteclassdiag}
\end{figure}

The implementation of the test suite structure can be seen in Figure \ref{fig:testsuiteclassdiag}.
The custom matrices are held as an array of Strings not of Matrix objects.
This is for no other reason than trying to recreate a test suite definition XML file in the form shown in Figure \ref{code:paulixtestset} requires the file name of the XML file specifying the custom unitary matricies.
If the array were an array of Matrix objects the file names would be lost or require an additional array.
The inclusion of a second array was discounted as it would introduce an unnecessary source of potential bugs concerning the two arrays being ``out of sync''.
This would mean the Matix objects may not represent the matrix encoded in the respective file as the Matrix objects are able to be modified and altered by the program without the file being updated.

Alongside the framework, an independant test suite graphical editor is provided.
A user can use this to produce the XML file and therefore do not need to explicitly create the XML file.
The design and implementation of the graphical editor can be found in Section \ref{sec:indtestsuiteeditor}.

The contents of a test suite are not fixed.
If a user finds an error or wishes to add test cases, either a third party application or the provided test suite graphical editor can be used to update the respective test suite elements.
These changes need to be reflected in the XML file to become persistant so to ensure that the updated XML file is well formed the framework provides a class to produce from a given test suite.
This class is used in the independant test suite graphical editor and is recommended for use by any third party application.

When editing a test suite, if a test set for $n$ qubits is added to the test suite and the test suite already contains a test set for $n$ qubits, the two test sets are merged.
The test cases of the test set being added have their ID and labels modified so that they continue the sequence of the existing test cases already in the test suite.
When the test sets have to be merged the test cases are cloned.
This is incase the test set is used in a different test suite.
If this were not done, when the ID and label were modified they would also be updated in the second test suite.
This would cause unpredicatable behaviour.

\subsection{Manager Classes}
\label{sec:manclasses}
As can be seen in the architecture diagram of the framework, Appendix \ref{sec:archdesign}, there are several classes with names suffixed with ``Manager''.
These classes provide access to the extendible areas of the framework.
There is a Manager class for the Fitness Functions, the Search Engines and the Search Problems.
Each of these are a specific site of expansion.

Each Manager is configured using an XML file specfying all options for the specific Manager.
The Fitness Function Manager will be configured for all the available Fitness Functions, the Search Engine Manager for all the available Search Engines, and the Problem manager for all the available Search Problems.
The Search Problem Manager class is outlined in detail in Section \ref{sec:problemman}.

This configuration is performed at runtime rather than at compile time.
It was designed to allow researchers to add, for example, extra Fitness Functions without altering the code of the framework.
This independence of the framework implementation and the results of research, specific Fitness Funcitons etc, has been identified as one of the key foundations of the framework concept.
The inclusion of this knowledge separation encourages the use of the standardised interfaces specified for each expansion site.

\lstset{language=XML}
\begin{figure}
\begin{lstlisting}
 <FitnessFunc>
	<FitnessFunctionTag>
	  <Name>FITNESS FUNCTION NAME</Name>
	  <Class>IMPLEMENTING FULLY QUALIFIED CLASS NAME</Class>
	  <Desc>FITNESS FUNCTION DESCRIPTION</Desc>
	</FitnessFunctionTag>
</FitnessFunc>
\end{lstlisting}
\caption{XML for Fitness Function Manager Configuration}
\label{code:fitfuntmanconfig}
\end{figure}

The XML outline shown in Figure \ref{code:fitfuntmanconfig} is an outline of the XML file used to specify the available Fitness Functions.
The XML files specifying the available Search Engines and the available Problems can be found in Appendix \ref{sec:semanspecxml} and \ref{sec:probmanspecxml}.

These XML files are used to regiser the available implementations with the respective Managers.
The Manager classes use these registrations to provide the choice of available instantiations of Search Engines and Fitness Functions.

\begin{figure}
\centering
\begin{emp}[classdiag](20, 20)
Class.E("XYTag")("-Name:String","-Description:String","-ClassName:String")();
Class.F("XYManager")()("+getAvailableXYs():Set<String>","+getXY(String key):XY","+getSearchEngineDesc(String key):String");
leftToRight(75)(E, F);
drawObjects(E, F);
clink(association)(E, F);
item(iAssoc)("1")(obj.nw = E.e);
item(iAssoc)("1..*")(obj.ne = F.w);
clink(association)(F, E);
\end{emp}
\caption{Manager - Tag Class Diagram}
\label{fig:mantagclassdiag}
\end{figure}

Both the Search Engine and Search Problem Manager classes are implemented in the same way.
The class diagram in Figure \ref{fig:mantagclassdiag} shows the Manager design.
A mapping between the Name of the registered options and Tag objects is held in each Manager.
A Tag contains the information within the \lstinline{<XYTag> </XYTag>} element of the XML file.
The getAvailableXYs() method returns a set of strings, the registered names, rather than the XY Tags to ensure the Manager classes have full control over the creation of the managed elements.
To provide an XY object when requested using thegetXY(key) method, the Manager classes use reflection to instantiate an object of the respective class, the value of the ClassName variable of the Tag registered with $Name=key$.

\subsection{Multiple Search Engines}
The framework is aimed to be used universally by Quantum Algorithm researchers.
The search techniques used by these researchers are also a matter of research effort.
If the tool were to provide a search engine, with no option for change, the use of the tool is likely to be significantly impacted.

Providing a simple interface that allows each researcher to potentially use a different search technique is likely to increase the tools applicability.
The simple interface allows a user to:
\begin{itemize}
 \item retreive the names, used as the search engine identifier, of all registered Search Engines
 \item retreive the instantiation of the specified Search Engine instantiation
 \item retreive the description of the specified Search Engine
\end{itemize}

All registered Search Engines must implement the supplied interface.
The Search Engines are not restricted to evolutionary approaches.
The internal workings of the different search engines are unrestricted.

The alternative approach would have been to implement a series of search engines based on several different techniques and provide researchers this choice.
This was not accepted as it moved the tool away from the framework intended.
The provision of search engines without a simple manner to add additional engines would restrict research and not allow researchers to easily use techniques developed in the research community within the system.

\subsection{Multiple Fitness Functions}
As was noted by Massey\cite{masseythesis}, different Fitness Functions can have a dramatic impact on the success of a Quantum Algorithm search.
The inclusion of a choice of Fitness Functions is to account for this.
As with Search Engines, the choice of Fitness Functions is provided by the Manager class through a simple interface with methods synonomous to those provided for the Search Engine selection.
A Fitness Function interface is provided to ensure that all Fitness Functions are able to be used universally within the tool and are not specific to any particular Search Engine for example.

Similarly to the Search Engine, a series of Fitness Functions could have been implemented and provided without provision for extension.
The justification for the approach taken is the same as listed for the multiple Search Engines.
It was deemed detrimental and in contradiction of the frameworks purpose to limit the Fitness Functions to those provided by the tool.


\subsection{Multiple Problems and Problem Specification}
\label{sec:problemman}
As has been mentioned on several occasions, one of the foundation principles of the framework is the ability to ``Plug and Play'' the work of other researchers without the proplem of integration.
With Search Engines and Fitness Functions developed to adhere to the respective interfaces, a user should be able to work with the toolkit and treat it as a ``Black Box''.

In Section \ref{sec:testsuitestruc} how test suites and all their contained test cases are specified in XML was described.
The use of XML files does however increase the effort required from the user.
The user needs to specify, each time they use the framework, the location of the XML file containing the correct test suite.
To reduce this effort the problem container is introduced alongside its manager.

The problem manager allows multiple problems to be defined within a single XML file so the user need not provide the test suite XML each time the framework is used.
This single XML file contains the definition of mutiple problems.
An example of these XML files can be seen in Figure \ref{code:probmanconfig}.
A problem has a name, description and file name for the respective test suite XML file.
The name and description are used to provide a human readible explanation of the problem represented by the test suite XML file.
The use of a separate XML file to collate all defined problems makes maintenance much simpler.

Providing a problem manager allows the framework to be used for different problems without having to restart the system and without any external software needing to provide different problems explicitly.

\lstset{language=XML}
\begin{figure}
\begin{lstlisting}
<Problems>
  <prob>
    <Name>Final Pauli X</Name>
    <DefFile>config/finalpaulix.xml</DefFile>
    <Desc>A Pauli X gate on the final Qubit</Desc>
  </prob>
</Problems>
\end{lstlisting}
\caption{XML for Problem Manager Configuration}
\label{code:probmanconfig}
\end{figure}

The Search Problem Manager works in a similar way to the Search Engine and Fitness Function Managers detailed in Section \ref{sec:manclasses}.
However, the difference between the registered Search Problems is the test suite configuration XML file rather than the implementing class.
A second difference is that the registered options are not static, a third party application can interact with the Manager to create new Search Problems and update the already registered Search Problems.
Therefore, the Search Problem Manager has to be able to produce an updated Search Problem Manager configuration file so the changes made are persistent.
The changes that the Seach Problem Manager is concerned with is only updates to the values in the Search Problem Manager configuration file.
Any changes made to the set of Test Cases are not maintained by the Search Problem Manager.
This ensures that the different concerns are separated.
The update of Test Cases is detailed in Section \ref{sec:testsuitestruc}.

\subsection{Quantum Algorithms}
\label{sec:quantalgs}
The result of the search engines are quantum algorithms.
To maintain the ``Plug and Play'' nature of the framework, the representation of these algorithms needed to be specifed and standardised.
However, the representation also had to ensure that it was not limiting the search engines.

To provide a standardised and non-limiting represenation the framework provides an internal quantum algorithm structure that can be simply built by any search engine.
This allows the search engines to have a different internal representation that is then used to build the standardised algorithm.
Using this there are no limitations on the structures used internal to the search engines.

The use of the standardised quantum algorithm also ensures that the reporting of an algorithm to the user is consistent.

\begin{figure}
\centering
 \begin{tabular}{|c|c|c|c|c|}
  \hline
Instruction & Gate1 & Gate2 & Phase & Sub-Algorithms \\
QuantumInstruction(Enumeration)&expnode&expnode&expnode&QuantumAlgorithm[]\\
\hline
 \end{tabular}
\caption{Quantum Instruction Structure}
\label{tab:quantinststruct}
\end{figure}

\begin{figure}
\centering
 \begin{tabular}{|c|c|}
\hline
$exp \rightarrow e + e$ & $e \rightarrow exp$ \\
$exp \rightarrow e - e$ &  $e \rightarrow SystemSize$ \\
$exp \rightarrow e * e$ &  $e \rightarrow Value$ \\
$exp \rightarrow \frac{e}{e}$ &   \\
$exp \rightarrow e^e$ &   \\
$exp \rightarrow \frac{\pi}{2^e}$ &   \\
$exp \rightarrow LoopVars[e]$ &   \\
\hline
 \end{tabular}
\caption{Expnode Context Free Grammar}
\label{tab:expnodecontext}
\end{figure}

An algorithm is a list of instructions.
The instructions that are used in the standardised quantum algorithms take the form shown in Figure \ref{tab:quantinststruct}.
The list of values that the Instruction element can take can be found in Appendix \ref{sec:alginstructionlist}.

$Gate1$, $Gate2$ and $Phase$ are all listed as being of type \emph{expnode}.
The final type for $Gate1$ and $Gate2$ is integer and for $Phase$ is double.
They are not explicitly these types to increase the expressive power of the algorithms.
The grammar that defines the \emph{expnode} type can be seen in Figure \ref{tab:expnodecontext}.
The \emph{expnode} type is needed so that the algorithms can react to the parameterisation that provides the increased power when compared to quantum circuits.
As the circuit size is a variable a constant cannot be used for $Gate1$, $Gate2$ or $Phase$ as the value may depend on the circuit size and have to be calculated when a circuit is being built using the algorithm.

\lstset{numbers=left,language=Java}
\begin{center}
\begin{tabular}{c}
\begin{lstlisting}
for(int i = 0; i < n ; i++){
  // You can use i here
  for(int j = 0; j < n ; j++){
    // You can use i and j here
  }
}
\end{lstlisting}
\end{tabular}
\end{center}

With the inclusion of an loop control construct in the form of the \emph{iterate} and \emph{reviterate} instructions additional parameters are introduced.
In Java and many programming languages, as can be seen in the example of Java code above, when using nested loops access to all loop variables is provided.
This means that at line 2 the loop variable $i$ is able to be used but at line 4 both loop variable $i$ and $j$ can be used.
This nesting is quite a powerful language feature.
With the inclusion of the iterate instructions and therefore the posibility of nested iterations this loop variable access could be implemented in one of two ways.
The most na\"{\i}ve implementation would be to just allow access to the ``closest'' loop variable.
This would mean that the code at line 2 wouldn't be effected but the code at line 4 would no longer be able to use the $i$ variable.
The more sophisticated option, that is implemented, is to provide access to all loop variables.
This is a second parameter that requires the use of the \emph{expnode} type.
All current iteration variables are provided in a \emph{LoopVars} array.
As can be seen in Figure \ref{tab:expnodecontext} the array is indexed by the value of a sub-expression.
To ensure that all indicies requested are valid the value of the sub-expression is calculated using modulus \emph{LoopVars.length()}.
If the \emph{LoopVars} array has a of length $0$ the result is $0$ irrespective of the index requested.

The difference between the \emph{iterate} and \emph{reviterate} instructions is that \emph{iterate} counts from $1$ to $Gate1$ and \emph{reviterate} counts from $Gate1$ to $1$.
As is explained in Section \ref{sec:qubitnum}, the qubits are numbered from $1$ to $n$ which is why the iteration instructions count to and from $1$ rather than $0$ as is normal in Computer Science.
The two different interate instructions are needed as they can express some looping constructs in a much simpler form that would be possible with just one of them.

For all \emph{Create\_*} instructions, $Gate1$ is used to index the qubit the actual gate should be assigned to.
Gate2 is only used by \emph{Create\_C*} and \emph{Create\_SWAP} instructions to index the control qubit and second qubit respectively.
For the \emph{Iteration} instruction, Gate1 is used for the number of iterations, $n$.
For the \emph{Create\_R*}, \emph{Create\_CR*}, \emph{Create\_P} and  \emph{Create\_CP} instructions, the Phase element is used to parameterise those gate to specify the amount of rotation applied by the resulting gate.


\subsection{Qubit Numbering}
\label{sec:qubitnum}
One of the major decisions made relating to the way in which the quantum algorithms are produced was that of which way the qubits should be numbered.
The two options were obviously in accending or decending order.

The chosen approach was the decending order.
This meant that for state $\ket{ab{\dots}st}$ the qubit represented by $a$ would always be given the identifier equal to the number of qubits in the system.
For example, if there were three qubits in the system the identifier of the qubit represented by $a$ would be $3$.
This was chosen to ensure that an identifier always represented the same qubit, irrespective of the number of qubits in the system.

The justification for this is to make the algorithm much more understandable.
If the identifers were dependant on the number of qubits it would make the results of the system much less comprehensible.

The use of this numbering is also much more natural as the identifer, $x$, of a qubit, $a$, is related to the value of the qubit when read in binary.
The value of the qubit $a$ is $2^{x-1}$.
This makes the optimisation of gate application, see Section \ref{sec:quantumgates}, much simpler.

\subsection{Quantum Circuits}
To perform the evaluation of an algorithm the circuits for the test sets need to be produced.
Both the representation of the circuit and the mechanism to construct the circuit from the algorithm needed to be standardised to ensure the ``Plug and Play'' nature of the framework.

The framework provides a default circuit builder.
The framework does allow a separate circuit builder to be provided as long as it conforms to the interface and the circuits it produces also adheres to the respective interface.
There is no manager class provided for circuit builders.
This was due to an assessment of the intended uses of the framework.
It is intended that the framework would be used primarily to perform the following:
\begin{itemize}
  \item Perform research into the effect of different fitness functions on the search for quantum algorithms
  \item Perform research into different search techniques that could be used to produce quantum algorithms
  \item Perform research to produce new quantum algorithms for a specific problem
\end{itemize}

It is not seen as a priority of the system to provide the same level of flexibility to the circuit building as the search engines and fitness functions.

The circuits that are produced by a circuit builder are hidden behind an interface.
This is to allow third party circuit builders to use their own internal representation and also to allow any future optimisations made in future work on this framework to be made without impacting the work of researchers.

The circuits produced provide the represented quantum circuit as an ordered iterator of quantum gates.
The use of an ordered iterator rather than a specific data structure is to ensure that any future optimisation or third party circuit representation is not limited.
It also reduced the potential errors involving the interpretation of a more complex data structure.

The circuits also provide a Latex representation to allow the circuit to be visualised.
The Latex representation uses the QCircuit package that can be freely obtained at \cite{QCsite}.

The circuit implementation, \emph{basiccircuit}, provided in the framework is implemented as a simple ordered list.
It uses the Java provided \emph{LinkedList} implementation.

\subsection{Quantum Gates}
\label{sec:quantumgates}
Any quantum circuit will be a series of quantum gates on specified qubits.
The quantum gates provided by the system are hidden behind an interface.
This is to ensure that any future optimisation of any gate's implementation cannot interfere with the implementation of any other component of the system.

Each quantum gate is required to provide a unitary matrix but it is not required that the matrix must be used in the application of the gate.
For quantum circuits with a high number of qubits, the cost of simulation increases rapidly.
This is mainly due to the increase in state vector and unitary matrix sizes.
Matrix multiplication is used to apply a unitary operation to a state vector, yet it is a very expensive operation.

To improve the performance optimisations can be applied for several gate types.
This is most obvious when analysing the operation of the Pauli X gate.
Figure \ref{eq:paulixcheaptrickvisual} shows, with the help of colour, that the application of a Pauli X gate on Qubit 1 is essentially a flip of neighbouring values.
This is also true for a Pauli X gate on any other qubit, just the definition of a state's ``neighbour'' is modifed with respect to the identifier of the qubit on which the gate is applied.

\begin{figure}
\[
\begin{tabular}{ r c l }
  \(\begin {pmatrix}
    \textcolor{blue}{a}\\
    \textcolor{red}{b}
  \end{pmatrix}\) 
& 
  \(\rightarrow\) 
& 
  \(\begin {pmatrix}
    \textcolor{red}{b}\\
    \textcolor{blue}{a}
  \end{pmatrix}\) \\

\\

  \(\begin {pmatrix}
    \textcolor{blue}{a}\\
    \textcolor{red}{b}\\
    \textcolor{green}{c}\\
    \textcolor{cyan}{d}
  \end{pmatrix}\)
  & \(\rightarrow\)
  & \(\begin {pmatrix}
    \textcolor{red}{b}\\
    \textcolor{blue}{a}\\
    \textcolor{cyan}{d}\\
    \textcolor{green}{c}
  \end{pmatrix}\) \\
\end{tabular}
\]
\label{eq:paulixcheaptrickvisual}
\caption{Visual Representation of Bit Manipulation Equivelent of Pauli X Operation on Qubit 1}
\end{figure}

The use of these tricks is not specified but the interface has been designed to ensure that the gate implementations can use such tricks or matrix multiplication interchangeably.
The interface includes an \emph{apply} method that takes an initial state and returns the state after the application of the gate.
This ensures that only the gate implementation needs to understand how the circuit should be applied.

Each gate must also provide a QCircuit represenation for use by the circuit to produce the QCircuit representation of the complete circuit.

The implementation of gates effecting two qubits are hidden by an extended interface to provide access to the identifier of the second qubit but ensures that all standard gate operations are also available.

\begin{figure}
\[
\begin{tabular}{ | c || c | }
\hline
\textbf{Matrix Manipulation} & \textbf{Bit Manipulation} \\ \hline
Phase Gate & Hadamard \\ \hline
RX Gate & Pauli X \\ \hline
RY Gate & Pauli Y \\ \hline
RZ Gate & Pauli Z \\ \hline
V Gate & \\ \hline
W Gate & \\ \hline
Custom U Gate & \\ \hline
Controlled U Gate & \\ \hline
Swap & \\ \hline
\end{tabular}
\]
\caption{Gate Implementation}
\label{fig:gateimps}
\end{figure}

As has been mentioned, a number of the gates shown in Figure \ref{fig:providedgates} can be implemented either as a matrix multiplication or a bit manipulation.
Several of the gate implementations in the framework use the bit manipulation and others use the matrix multiplication.
The table in Figure \ref{fig:gateimps} shows which of the provided gates are implemented using the matrix manipulation and which use the bit manipulation.
For all gates that use matrix multiplication their equivalent unitary is based on those listed in Figure \ref{fig:providedgates} but has to adapt to the qubit they are applied to and the system size.
The equivalent matrix is calculated in two phases.
The initial phase is the constuction of the matrix listed in Figure \ref{fig:providedgates}.
This initial phase is calculated in the constructor of each gate.
The gate on which the gate is applied is provided as an argument to the constructor of the gate.
Unfortunately a Java interface cannot specify the signiture of the contructor of implementing classes and abstract classes require the constructor to be fully defined which is not possible when the constructor needs to perform this first phase.
As a result this cannot be ensured but is expected as part of implementing the interface.
The second phase is the combination of the matrix constructed in the initial phase and identity matrices for all qubits not effected by the gate.
This second phase is performed as part of the \emph{apply} method.

There are several gate implemetations that deserve a more detailed summary.
To produce the unitary matricies shown in Figure \ref{fig:providedgates} for the RX, RY and RZ gates the matrices of other gates are used.
Equations \ref{eq:rxcalc} - \ref{eq:rzcalc} shows the details of these matrix calculations.
These equations are taken directly from Lecture 9 of \cite{QIPLect}.
\begin{eqnarray}
\label{eq:rxcalc}
 R_x(\theta)&=&\cos{\frac{\theta}{2}}I-i\sin{\frac{\theta}{2}}X \\
\label{eq:rycalc}
 R_y(\theta)&=&\cos{\frac{\theta}{2}}I-i\sin{\frac{\theta}{2}}Y \\
\label{eq:rzcalc}
 R_z(\theta)&=&\cos{\frac{\theta}{2}}I-i\sin{\frac{\theta}{2}}Z 
\end{eqnarray}

The second gate implementation that requires a more detailed summary is the Controlled U Gate.
The implementation of the gates cannot assume that the circuit that is produced by an algorithm is correct.
Placing all the restrictions on the search process that need to be placed on the circuit would have a serious impact on certain search processes.
Therefore all the restictions need to be handled by elements of the framework.
The restriction for Controlled U Gates is that the control qubit cannot be a qubit that is effected by the U operation.
If the specifed qubit would be affected by the U operation the implementation acts as though it was not controlled.
A second consideration for the design and implementation is if U is a custom gate the U matrix can change.
This means that the calculation of the matrix that represents the controlled version of U needs to be able to update depending on the test case.

A major implementation detail is how the matrix that represents a controlled version of unitary U is calculated.
The calculation changes depending on the relative position of control qubit.
If the control qubit is a higher significant qubit the calculation is different than if the control bit is a lower significant qubit.
Equation \ref{eq:clu} is the calculation if the control qubit is a lower significant qubit.
Equation \ref{eq:chu} is the calculationif the control qubit is a higher significant qubit.
$S1$ is an identity matrix of the size $2^n$ where $n$ is the number of lower significant qubits that aren't effected by the Controled U gate, a control bit or between the conrtol qubits and effected gates.
$S2$ is an identity matrix of the size $2^n$ where $n$ is the number of higher significant qubits that aren't effected by the Controled U gate, a control bit or between the conrtol qubits and effected gates.
U is the matrix that defines the operation of the gate under control.
This unitary matrix is retrieved using the \emph{getUnitary} operation on an object of the correct class.
The \emph{getUnitary} operation returns the matrix created by the first phase explained previously.

\begin{eqnarray}
\label{eq:clu}
 CU&=& (\begin{pmatrix}1&0\\0&0\end{pmatrix})\otimes S1 + \begin{pmatrix}0&0\\0&1\end{pmatrix}\otimes U) \otimes S2 \\
\label{eq:chu}
 CU&=& ( S1 \otimes \begin{pmatrix}1&0\\0&0\end{pmatrix}) +U \otimes \begin{pmatrix}0&0\\0&1\end{pmatrix}) \otimes S2
\end{eqnarray}

Equations \ref{eq:clu} and \ref{eq:chu} are generalisations of the Controlled U gate that have not previously been published.

As with standard gates, the calculation of the full matrix is performed in two stages as previously decribed in this section.
If U is a Custom gate then the full matrix is performed in a single stage as described in Section \ref{sec:custgates}.

\subsection{Custom Gates}
\label{sec:custgates}

As has been explained in Section \ref{sec:testsuitestruc}, the framework has to provide a way to include custom gates.
The way of specifying the custom unitary matrices is explained fully in Section \ref{sec:testsuitestruc}.
To use these custom unitary matrices, the framework provides a gate implementation called \emph{Custom\_Gate}.
This is significantly different from all other gates in one particular feature.
All gates listed in Figure \ref{fig:providedgates} have a fixed unitary matrix and only have to adapt for the number of higher significant qubits that are not effected by the gate.
This is not so for custom gates, their operation change on a test case by test case basis and therefore both the ``phases'' mentioned in Secion \ref{sec:quantumgates} are performed during the \emph{apply} method.

The \emph{apply} method in the Gate interface has two parameters, one is the quantum state on which the custom operation is applied and one that is the test case.
The test case is provided to extract the array of matrx definition file names.
An alternative would be to pass an array of the Matrx objects encoded in the files refered to in the test case.
The custom matricies are held in the test case objects an array of strings to ensure that the information about which file encodes the matrix is not lost.
The decision to pass in the test case rather than an array of Matrix objects or even the file name strings is for the same reason.
If an array of matrices were passed in then it would be possible for the matrices to be inconsistent with the matrices encoded in the files specifed in the test case.
If an array of file names, strings, were passes as an argument then it would be possible for the file names to be inconsistent with the file names specified in the test case.

Using the test case that is supplied as an argument, the \emph{apply} method uses an argument provided to the constructor of the gate as in index into the array of file names held in the test case.
The file is decoded to produce the unitary matrix, using this the full matrix, including identity padding for uneffected qubits, is produced and multiplied with the state.
This could be seen as slightly wasteful requiring the matricies to be created from file each time but it was decided that the integrity of the evaluation, ensuring the matrix used is that encoded in the file, is of higher importance than a slight increase in efficiency.

The \emph{getUnitary} method also uses the test case provided to retrieve the matrix from file using the file name stored in the test case rather than an array of file names or Matrix objects for the same reasons as the \emph{apply} method.

\subsection{Search Engine Parameters}
With many search techniques there are a number of parameters that can be configured and altered, sometimes having dramatic effects on the search results.
To enable the configuration of these parameters through the framework a suitable, and flexible method had to be introduced into the design.
Not all search techniques have the same number of parameters to configure and due to search engines being developed potentially by different researchers, even if the parameters are the same, their internal representation may be different.

The framework requires a search engine to provide two \emph{search} methods, the initiator of the algorithm search process using current settings.
The first takes two arguments.
The first argument is a boolean array.
This is an array with a length equal to the number of gates in the \emph{QuantumInstructionEnum} enumeration.
The value at each index indicates whether the element of the \emph{QuantumInstructionEnum} enumeration with the same index should be included in the search and therefore allowed in the result.
The second is a Object array.
This second array is provided to allow search engine researchers to have search engine parameters that are configured by a third party application.
It is not possible to use to make the array more specific without restricting the parameters available for the researchers to provide.
It does place the responsibility on the researcher to provide adequate documentation and explicit error messages with regards to the ordering and the type of the parameters required by the search engine.

The second takes no arguments and is intended for search engines that require no parameters and search engines that provide their own interface to collect the parameters.
The documentation provided by researchers for each search engine must specify which of the two options applies to each search engine.

If a search engine comes under the second category and is providing a graphical user interface, the system must be run on a system configured to allow GUIs to be displayed.
The configuration of the user interface is controlled entirely by the search engine rather, than any third party software or the framework.
This option was chosen as it is assumed that the person with best understanding of the parameters provided by a search engine is the same person who implemented the search engine.
This is to improve the user interface with respect to the structure principle, see Section \ref{sec:provgui}.
The ordering of parameters in the Object array of the first \emph{search} method cannot be assumed to be a good indication of grouping and there is no indicated separation of groups.
As a result if the framework, or a third party client, were to provide an the user interface, semantically linked parameters may not be grouped by the user interface.

A second justification for this based on the ``Plug and Play'' nature that is a foundataion of the framework.
If the configuration were provided by third party software providing a GUI for the framework, or embedding it in a larger software suite, the third party software would have to be adapted for each search engine implementation or automatically generated.
Automatic generation is unlikely to produce a ``principled'' user interface, see Section \ref{sec:provgui}.
Framework and third party client modification for each search engine is not acceptable and is totally incompatible the motivation behind the framework.
With such a design it is likely that the research community would see no advantage to using the framwork than working as they are now, in complete isolation.

\subsection{Design as a Black Box}
The framework is design to be used as a black box.
All external code is only allowed access through the use of interfaces and final or abstract class types.
The implementation of the internal framework functionality is hidden and not required by third party software.
This desgin ensures that third party software shall be loosely coupled with the framework.

Likewise, all internal modules (algorithms, circuit builders, circuits, gates, circuit evaluations, etc) are also designed to view each other as black box modules.
This improves the modularity and reduces the coupling between these internal modules.

\subsection{Algorithm and Circuit Evaluation}
\label{sec:cireval}
When an algorithm is produced by a search engine, the circuit is built for each test set.
This circuit then needs to be evaluated against all test cases held in the respective test set.
As mentioned in Section \ref{sec:testsuitestruc} the framework uses a start and final state pair to define the desired operation of the algorithm.
Therefore the framework must provide the simulation of an arbitary circuit on and arbitary start state.

Due to the design decisions taken previously all the functionality is provided and just needs a simple management process.
With the circuit interface providing an ordered iterator over the gates in the circuit, and each gate providing an \emph{apply} method the management just needs to work through all the gates in the iterator and calling the \emph{apply} method with the returned state of the previous call.
To evaluate the circuit against a test case, the process simply needs to use the start state provided by the test case as the argument of the \emph{apply} method of the first gate.
The state returned by the final call to the \emph{apply} method is a state equivalent to the quantum state that would exist if the circuit were to be produced and provided with the same initial state.

The suitability measures have a very simple interface.
They only provide a method to retrieve the name of the suitability measure and a method to produce a numerical measure of similarity between two quantum states.

Using the selected suitability measure  the final state produced by the management process is compared to the expected final state defined in the test case.
Repeating this process for all available test cases in the current test suite, a numerical and therefore comparable suitability value can be assigned to any algorithm produced by the search engine.

This is one example where the design of the framework ensures that the when combining all the elements of the framework (the gate, the circuit, the test cases etc) the logic required is simple.
This improves the readability of the code and therefore is likely to improve the quality of the code.
With higher readability the number of software bugs are likely to be reduced due to higher levels of understanding.

It is assumed that when a researcher creates a test suite, the test cases that are included are those that the have a higher interest in.
As a results it can also be assumed that the researcher would be interested in the final state produced by the best solution found by the search engine for all the start states provided by the test suite.
The circuit evaluator interface provides a simple method call to produce these.
The \emph{getResults} method returns a test suite data structure that is almost identical to that provided by the search problem used in the search.
The only difference is that the final state of each test case in the returned structure is not the desired theoretical state but the actual final quantum state produced by the circuit simulation.

\subsection{Step-By-Step Evaluation}
The framework is required to provide a step-by-step evaluation facility.
There are two abstract ways this can be provided.
The framework can provide an interactive process that only applies a gate when it receives the command to do so and reporting the ``current state''.
The second option is to record the ``current state'' at each point in the circuit.

The main advantage of the first method is that it doesn't require processing of gate applications if the interactive evaluation does not reach them.
A second advantage is that the memory requirement is very small as only the ``current state'' and the position in the circuit the evaluation is up to.
A disadvantage is that for complex gates the application of the unitary operation could take longer than it is acceptable for a user to wait.
A second disadvantage is that if the memory required is kept to the minimum, moving backward will also require the application of a gate.
However, the gate that needs to be applied is not necessarily the same gate that is applied when working forward.
This could make the step-by-step evaluation quite complex to implement.
The time required for the reversed gate to be calculated and applied could also be an issue for responsiveness.

The main advantage of the second alternative is that to move forward, and also to move backward, is a simple loading of the respective ``current state'' from a data structure.
This results in the requests from the user to move forward and backward in the circuit taking approximately constant time irrespective of the steped over gate's complexity.
A second advantage is that taking a step backward in the evaluation never requires the inverse of a gate to be calculated.
For gates such as the Pauli gates this is not an issue as they are their own inverse.
However this is not the case for arbitrary gates requiring the inverse of aribitary matricies to be calculated.
The main disadvantage of the trace method is that the memory requirements are not constant, they are linear with respect to the number of gates in the circuit but exponentially with respect to the number of qubits in the system.
The second disadvantage is that although stepping through a circuit does not require the application of gates to the ``current state'', the full trace has to be produced before the step-by-step evaluation can be performed.
This means that if the circuit produced is large and complex, the start up and initialisation time could be significant.
The requirement for the full circuit to be evaluated irrespective of how many steps are taken in the step-by-step evaluation could lead to trace elements never being reached.

The way the framework has been designed is to include the second of these two options.
The additional memory requirements are not likely to be excessive when viewed in respect to the amount of RAM available in the average PC.
The main reason for the choice was that the calculation of all ``current state''s is done once and can be accessed as many times without extra computation required.
This should make the reponsiveness if the framework much better.

There is also a second major design decision to be made with respect to the step-by-step evaluation, what the initial state should be to produce the trace.
There are really two alternatives.
The initial state could either be provided by the user explicitly or the test cases of the search problem could be used as the source of the initial  states.

The decision was made to combine the two options.
The framework accepts as an input to the step-by-step tracer a test suite data structure.
The test suite can either be the same that was provided by the Search Problem and used by the suitability measure or can be a new test suite that has been created.
This allows the client or third party application to provide either or both of the options.

The circuit evaluator interface provides the \emph{getTrace} method that provides a list of test sets.
The step-by-step evaluation is performed on a circuit rather than an algorithm.
This means that only test cases for a specific number of qubits can be stepped through together, this is why the list is of test sets rather than test suites.
It would have been possible to produce the traces for all test sets in the test suite and then returned a list of test suites.
This was not implemented as it is likely that a reasearcher would try and understand each circuit in turn and then look for how they related as a second phase.
The step-by-step evaluation is expected to aid in the understanding of the circuits rather than the comparisions.
Therefore if a researcher had produced a large test suite, and is only concerned with the understanding of an imparticular circuit, the wait for the trace to be produced for the full test suite is likely to be significantly longer than for just the relevant test set.

The implementation of the circuit evaluator that provides this trace runs through each test case in turn.
After each gate has been applied, the ``current state'' is set as the final state attribute of a cloned version of the current test case that is subsequently added to the returned data structure.
The implementation is purely concerned with providing the trace, not suitability measure evaluation is performed.

As mentioned in Section \ref{sec:cireval}, the design of the framework makes the creation of this trace much simpler.
The separation of concerns makes the tracing logic much simpler and much less obscured than it would have been if design decisions regarding the gate and circuit interfaces had not been made.

\subsection{Batch and Distributed Processing}
The framework produced is not expected to be used solely for Quantum Algorithm research.
It is also expected to be used for research into the search techniques themselves.
Therefore, the framework had to provide a managed solution to perform multiple iterations, a batch, and for the results to be collated for statistical analysis.

The search engine interface provides a \emph{getResults} method that returns an array of search results produced by the last call to a \emph{search} method.
The \emph{search} methods do not return the array themselves so the method is not a blocking call.
To ensure clients are informed when the results are available, the search engine requires any client to register as an observer.
This decision was taken so that clients did not have to provide the non-blocking mechanism.

The framework has been designed so as not to provide any additional help for batch processing for a very simple reason.
The way that batch processing is carried out is not constrained.
This allows the researcher to select how each iteration should be processed and what statistics need to be collected at which stage.
If the framework were to fully manage the batch operation researchers would be restricted.

For example, the framework could be implemented to sequentially execute each iteration.
For a researcher who has access to a cluster this could be highly frustrating.
Due to the complete separation of each iteration, there is no inter-iteration communication, it is perfectly suited to distribution.
It is effectively a Monte Carlo experiment.
However, if the framework were to provide a distributed computing mechanism, the researchers would be forced to use this.
This would mean that even if the chosen framework would be highly inefficient for a specific search technique or even if it was highly inefficient when running in a local mode there would not be any option available to the researcher other than to modify the framework.
This is obviously not desirable from a ``Plug and Play'' perspective.

This does however have the impact that different researchers could use different distribution frameworks.
If they are fully managed by the search engine this is not a problem but if they require additional configuration it does detract form the ``Plug and Play'' nature of the framework.
As a result a recommendation is placed on all search engines produced for the framework.
It is simply that a local version should also be provided.
This means that if a reasearcher wishes to use the search engine on a single machine or on an incorrectly configured cluster the application can still proceed.
This is likely to have an impact on performance when compared to it the same search being performed on a correctly configured cluster.

The prescription of a batch processing manager may also restrict the statistics that are required by the researcher.
As a result of the reporting of statistcs is also not provided by the framework but expected of the Seach Engine.

Both of these decisions may appear to reduce the power of the framework and the reasons to use it.
However, it is easy to argue that the management of batch distribution and statistical reporting would increase the power of the framework but seriously undermine the usability of the framework and therefore is unlikely to be adpoted by researchers.

\section{Provided Tools}
The research framework is accompanied with two independant tools.
These are not intended to take the place of a third party client.

All of the configurations for manager classes and encodings of stored test suites and matrices are stored in XML.
The configuration files for the manager classes are reletively simple and are unlikely to be hard to produce manually.
However, the encoded test suites and matrices are likely to require longer XML files.
To improve the efficiency of the researcher and to reduce the probability of simple, but hard to spot, errors being introduced graphical editors are provided alongside the framework.

\subsection{Graphical Test Suite Editor}
\label{sec:indtestsuiteeditor}
A meaningful test set is likely to contain at least 4 test cases, but usually many more.
Say for example that the 4 test cases are all for the system syze of 2.
This results in 4 test cases, each with 2 matrices each containing 4 complex numbers.
This would require an XML file to contain 64 floating point numbers, as complex numbers are stored in the XML file as two floating point values.
Remember that this is a test suite of just 4 test cases acting on 2 qubits, the number quickly increases with the system syze and number of test cases.
Writing this XML file by hand in a text editor is not only tedious, but error prone.

To reduce the tedium of the process and the number of errors a graphical user interface is provided.
The interface is simple and clear, not providing the user with anything that is not necessary.
The interface allows the user to either edit a test suite that is already encoded in a file, or to create a new test suite from scratch for up to 10 qubits.
The limit is not imposed by the framework but by the editor.
Any test suite containing test cases of greater than ten qubits, the final state is likely to have been calculated by a thrid party application and manual input of $2^{11}$ complex numbers for both a starting and final state is likely to introduce errors.
Therefore it is recommended that time is spent integrating the third party software and the framework so the test suite can be automatically generated.

A user can add and remove test sets and test cases.
Each test set is provided on a separate tab.
The state is represented by a $2^n\times{1}$ matrix.
This makes the simple tabular visualisation used to edit the starting and final states the natural choice.
Two tables are used, one to edit the starting state and one to edit the final state.
The complex numbers are not listed as two separate floating point values but in cartesian form.
When a user has created or emade the changes, the test suite can be saved to an XML file.
The XML file is one that matches the format defined in Section \ref{sec:problemman} and also contains comments to try and improve the understanding if it were to be read using a simple text editor.

The design of the editor follows the user interface design principles, see Section \ref{sec:provgui}.
The editor is very simple with no extraneous functionality.
The use of tabs ensures that the editor design is structured, all related test cases are shown together and all user controls are also grouped whether they act on the test suite as a whole or the current test set and test case.
The two tables use the same class and each tab is also of the same class ensuring consistent visual representation and component reuse.

\subsection{Graphical Matrix Editor}
\label{sec:indmatrixeditor}
As with the number of floating point values required to define a test suite, the number required to define a matrix rapidly grows with the number of qubits.
This makes the manual production of the defining XML files again tedious and error prone.
The matrix encoding in file is optimised so that only non-zero values are stored in the file potentially reducing the number of values required.
This improves the process slightly but is can still be seen as rather tedious and error prone.

The framework provides a very simple editor that can be used to create the XML files in a much more familiar way.
The editor allows users to edit a matrix currently encoded in a file or to create a matrix of up to four qubits from scratch.
This is not a limit imposed by the framework.
It is to try and encourage the integration with third party applications.
For gates acting on four qubits the matrix contains $256$ complex numbers, for gates acting on five qubits this increases to $1024$.
Manually inputing $1024$ complex numbers into a table with $32$ columns and $32$ rows makes it very easy to make a mistake.
For operations of this size it is likely the matrix would have been calculated by a third party application.
This limit on the editor is to try and increase the integration with third party applications and therefore reduce errors.

\section{Search Engine Implementations}
\subsection{Q-Pace IV Based Search Engine}
Alongside the framework, a search engine based in the Q-Pace IV search engine featured in \cite{masseythesis} has been developed.
This is a Genetic Programming based search engine using the ECJ\cite{ecjtool} system.

The search engine is provided in both local and distributed versions.
The local verision uses separate threads to serially perform interations.
The distributed version uses the JPPF Framework to perform iterations in parallel using all resources available.
To ensure that the function of both the local and distributed versions are the same, the two versions are based on the same code parameterised by the container class for the processing, Thread or JPPFJob.
The processing unit is implemented as a search engine core that implements the JPPFTask interface, which inherits the Runnable interface.
This allows the processing unit to be used both for the local and distributed version ensuring consistency.

The JPPF Framework was chosen for the distribution as it is very simple to configure and provides the functionality required in a very easy to userstand way ensuring code readability is maintained.
There are many other frameworks available as well as producing a bespoke distribution system.
The bespoke solution was discarded as it is completely opposed to the philosophy behind the framework.
The other open source frameworks, including Apache Hadoop\cite{apahadoop}, were not chosen as they provided much more functionality than was required and as a result were much more complex and intrusive in the source code.


\section{Suitability Measure Implementations}
\label{sec:provsuitmeas}
\subsection{Simple Suitability Measure}

\subsection{Phase Aware Suitability Measure}

\subsection{Simple Parsimonious Suitability Measure}

\section{Provided GUI}
\label{sec:provgui}
To design the provided GUI I followed the following principles:
\begin{itemize}
 \item \textbf{The structure principle:} Design should organize the user interface purposefully, in meaningful and useful ways based on clear, consistent models that are apparent and recognizable to users, putting related things together and separating unrelated things, differentiating dissimilar things and making similar things resemble one another. 
The structure principle is concerned with overall user interface architecture.
\item \textbf{The simplicity principle:} The design should make simple, common tasks easy, communicating clearly and simply in the user's own language, and providing good shortcuts that are meaningfully related to longer procedures.
\item \textbf{The visibility principle:} The design should make all needed options and materials for a given task visible without distracting the user with extraneous or redundant information. 
Good designs don't overwhelm users with alternatives or confuse with unneeded information.
\item \textbf{The feedback principle:} The design should keep users informed of actions or interpretations, changes of state or condition, and errors or exceptions that are relevant and of interest to the user through clear, concise, and unambiguous language familiar to users.
\item \textbf{The tolerance principle:} The design should be flexible and tolerant, reducing the cost of mistakes and misuse by allowing undoing and redoing, while also preventing errors wherever possible by tolerating varied inputs and sequences and by interpreting all reasonable actions.
\item \textbf{The reuse principle:} The design should reuse internal and external components and behaviors, maintaining consistency with purpose rather than merely arbitrary consistency, thus reducing the need for users to rethink and remember.
\end{itemize}
Taken directly from REFERENCE.
Where additional principles are used they are explained alongside the design element they refer to.

The design of the main screen of the GUI can be seen in Figure \ref{fig:MainGUIDesign}.
The figure shows the main screen after a a search has been completed.

\begin{figure}
 \includegraphics[width=\textwidth]{GUIDesign.png}
\caption{Main User Interface - After Search}
\label{fig:MainGUIDesign}
\end{figure}

Each section of the GUI is explained separately with reference to the principles listed.

\subsection{Main Window Layout}
As can be seen in Figure \ref{fig:MainGUIDesign} the layout of the main window separates the ``dissimilar things'' with the use of visible but subtle borders.
This is a result of both the \emph{structure} principle.
The interface is structured so that the centre of the display contains the information of the highest importance, strafed by two control menus.
This central panel collates all of the main results of the latest search.
% The search statistics are shown in the right hand menu area.
% This separation ensures that the GUI does not become cluttered.

The layout is also intended to take into account the recent move towards wide screen monitors.
Wide screen monitors provide a new problem in GUI design.
If a GUI fills the screen area and fills it fully with the display of information, it can appear stretched and distorted.
If the GUI were to have a single menu along one side it also doesn't look ``right'', it looks excessively heavy on the non-menu side.
This is a issue with standard monitors also but in my opinion is exacerbated by the wide screen ratios.
Although it isn't directly related to any of the design principles it is in my opinion an important property of a GUI to appear well balanced across the available screen area.

With the two menu panels the separation of the configuration options allows for a simple layout of configuration.
On the main window the only configuration that is available is the selection of the Search Engine, Suitability Measure and Search Problem.
The configuration of the Search Engine and the definition of the Search Problem is not handled by the main window.
This is to ensure that the display does not become overly cluttered, it maintains a simple and clean appearance.
This is a result of both the \emph{simplicity} and the \emph{visibility} principles.

\subsection{Search Engine, Suitability Measure and Search Problem Selection}

Before a search can be started, selections have to be made for the Search Engine, Suitability Measure and Search Problem.
The selection of these are provided by three drop down lists.
The available options for a user to select are limited to the Search Engines and Suitability Measures that are registered in the respective managers in the framework.
For a user to add a new Search Engine or Suitability Measure the respective XML configuration file needs to be updated.
This was done so as to follow the \emph{simplicity} and \emph{tolerance} principles.
It ensures that any selection made by the user is a valid selection.

The selection of the Search Problem is again provided by a drop down list, following the \emph{simplicity} and \emph{tolerance} principles.
The difference is that the creation of a new problem from scratch and using a predefined test suite held in an XML file can be performed within the GUI.
Despite this difference, the use of the drop down list still ensures that any selection made by the user is a valid selection.
The validity of each individual Search Problem is checked by the editor dialogs described next.

All three of these selections are performed in the same way and ensure that the GUI maintains a level of consistency.
When a selection is made, the selection is shown and a description, provided in the configuration XML files, is shown.
The combination of the description and the consistency were included for the \emph{reuse} and \emph{feedback} principles.

\subsection{Search Problem Creator and Editor}
As mentioned above, the creation of Search Problems is provided by an on-screen dialog.
The creation and editing of Search Problems is also provided by a standalone application, see Section \ref{sec:indtestsuiteeditor}.

The integrated editor and the standalone application use the same components and overall design.
The same components are also used when creating a new Search Problem and when editing an existing Search Problem.

Not only are the individual components reused but each dialog follows the same design layout.
This promotes the \emph{reuse} and \emph{structure} principles.

The dialogs ensure that the user has entered values for the required fields and ensures that each entry is valid.
This implicitly ensures that each selection available to the user in the Search Problem drop down list is a valid option.
This follows the \emph{Tolerance} principle.

\subsection{Reporting Results}

\begin{figure}
 \includegraphics[width=\textwidth]{GUIDesignProgress.png}
\caption{Main User Interface - Before Search}
\label{fig:MainGUIDesignProg}
\end{figure}
\begin{figure}
 \includegraphics[width=\textwidth]{AccurateReadOutMouseOver.png}
\caption{Accurate State Readout}
\label{fig:AccStateReadOut}
\end{figure}
When a Search Problem is selected, in the central area a visual representation of the test suite is produced.
This representation can be seen in Figure \ref{fig:MainGUIDesignProg}.

After a search has been completed, the final states produced by the realised algorithm are shown using the same representation.
This can be seen in Figure \ref{fig:MainGUIDesign}.
Using a simple visualisation like this makes the comparison between ``desired'' final states and the final state produced by the algorithm found by the search.
It is true that the visualisation could be too simple for small differences to be noticed.
To counter this problem the visualisation allows the user to hover the mouse over each column to get an actual value.
This can be seen in Figure \ref{fig:AccStateReadOut}.

This provides users with both a quick, simple and visual way to compare final states as well as an accurate way to compare final states.
The accurate comparison method provided was implemented rather than a value table as part of following the \emph{simplicity} principle.

The result of a search is the quantum algorithm found by the search, rather than the final states for the test cases described so far.
The GUI provides the user 3 different ways to see the search result.
A simple textual listing of the algorithm is provided in the same format as the framework produces on its own.
To help with the users understanding of what the algorithm means a circuit diagram is produced for a user controlled number of Qubits.
The diagram is produced using the symbols that are shown in Figure \ref{fig:providedgates}.
The circuits produced are not just provided as a circuit diagram but also in QCircuit representation so that the circuits can be placed into any publication produced in Latex simply with the use of the QCircuit package.

Providing the three representations meet the \emph{simplicity}, \emph{feedback} and \emph{reuse} principles.
The \emph{simplicity} principle is met as the result of the search is simplified to a human readable algorithm and circuits can be created to help the user understand how the algorithm is working.
The \emph{feedback} principle is met as the circuit diagrams that are produced are done so using widely accepted symbols and conventions for quantum circuit drawing.
The \emph{reuse} principle is met with the use of QCircuit to produce circuits in a form that can be included in publications.
An alternative would have been to output the circuit diagram that is drawn by the GUI as an image that could have been included in any, not just Latex, publication.
I feel the use of QCircuit is a better choice as the user then has control and is able to carry out, if necessary, manual circuit optimisation.

\subsection{Step-By-Step Evaluator}


\begin{figure}
 \includegraphics[width=\textwidth]{StepByStepEval.jpg}
\caption{Step-By-Step Evaluation Dialog}
\label{fig:StepByStepEval}
\end{figure}
The way in which quantum algorithms and the circuits they produce work is usually subtle and hard to understand by simply looking at the circuit.
The framework provides a step-by-step evaluation trace when provided with the input states.
The input states are provided to the framework in a test suite structure.
It was decided that the provided GUI would provide the test suite of the respective Search Problem.
This means that an evaluation trace is produced for each of the test cases, in each of the test suites.

The step-by-step evaluator is provided in a dialog rather than integrated in the main frame, this dialog can be seen in Figure \ref{fig:StepByStepEval}.
This was done so as to focus the users attention and to ensure that the addition of the functionality did not result in a cluttered GUI.
This follows the \emph{structure} and \emph{visibility} principles.

The step-by-step evaluation is performed with respect to a produced circuit.
This requires the user to select the number of qubits the circuit should be produced for and the step-by-step evaluation is provided for all test cases of that number of qubits.
Due to the design decision made for the framework, to provide a full trace rather than interactive evaluation, the test cases can be switched between at any step without returning to the start of the circuit.

The dialog provides a circuit diagram, an initial state selector and a visual representation of the state at the ``current step'' in the evaluation for the selected initial state.
The circuit diagram is produced by reusing the circuit diagram drawn in the results pane of the main window.
This ensure that the circuit is represented using the same standards as that shown in the results pane.
The only difference is that the ``current step'' is indicated using a vertical line on the circuit diagram, this can be seen in Figure \ref{fig:StepByStepEval}.
This follows the \emph{reuse}, \emph{feedback},  \emph{visibility} and \emph{simplicity} principles.

The visual representation and initial state selector are the same as those used to report the final states produced for the test suite in the main window.
The only difference is that only the test cases for the current number of qubits is shown.
The use of the same visual representation ensure the user does not have to understand anything extra to use this functionality and follows the \emph{reuse} principle.

% \subsection{Custom Gate Editor}



% \subsubsection{The Structure Principle}
% As can be seen in Figure \ref{fig:MainGUIDesign} the layout separates the ``dissimilar things'' with the use of headings and borders.
% The interface can be viewed as structued so that the centre of the display contains the information of the highest importance, straffed by two control menus.
% This central panel collates all of the result information regarding the result of the latest search.
% The search statistics are shown in the right hand menu area.
% This separation ensures that the GUI does not become cluttered.
% 
% This layout is also intended to take into account the recent move towards wide screen monitors.
% Wide screen monitors provide a new problem in GUI design.
% If a GUI fills the screen area and fills it fully with the display of information, it can appear stretched and distorted.
% If the GUI were to have a single menu along one side it also doesn't look ``right'', it looks excessively heavy on the non-menu side.
% This is a issue with standard monitors also but in my opinion is excasserbated by the wide screen ratios.
% 
% As well as the main GUI, the dialogs used to Create, Load and Edit search problems are all structured in such a way so as to guide a user through the respective operation.
% 
% The implemented search engine also provides a dialog box to configure a series of parameters.
% This dialog is split into two sections.
% One section contains a selection of all available gates while the second provides a series of GP parameters that the user may be interested in configuring, such as the number of generation and the population size.
% This separation follows the structure principle and clearly separates to the two distinct sets of configuration options.
% 
% \subsubsection{The Simplicity Principle}
% The way in which the framework has been designed aids in making the user interface simple.
% The framework requires very little configuration.
% The only real configuration that is required is the selection of which search engine and suitability measure to use and what problem you want to try solve.
% These selections are provided in very simple drop down lists for the users seletion.
% 
% The GUI provides a visual way to Create new and Edit existing search problems.
% A standalone editor is also provided.
% The editor provides a 
% 
% The only other two aspects of the configuration are the problem editor and the configuration of the search parameters.
% These are justified separately.
% 
% \subsubsection{The Visibility Principle}
% I think it is quite clear to see that the number of ``options'' visible to the user at any one time is minimised.
% The use of drop down boxes for selection naturally hides unwanted options until required.
% 
% In the central area the available selections are visible but are also much more likely to be ``browsed'' by the user.
% The decision not to use drop down boxes for the initial state selection is due to the expected ``browsing'' of this data.
% I expect that the user is likely to flick through the various options available which is much easier when the options are always visible.
% The use of a drop down box would also have required two mouse clicks per selection, one to open the list and one to make the selection.
% With the options visible this is reduced to a single click.
% 
% \subsubsection{The Feedback Principle}
% Once a selection is made for any of the selections available to the user the information on the display is updated accordingly.
% For example, the selection of a different search engine will update the contents of the search engine description area, selecting a different ``input state'' in the graph panels automatically updates the graph and the selection is highlighted.
% 
% 
% The user interface has been designed so that one user could leave it, a second user come to use it and understand very quickly the decisions that had already been made.
% It is not just the selection choices that are provided to the user as feedback.
% As can be seen in Figure \ref{fig:MainGUIDesignProg}, a progress bar is provided to indicate to the user how far through the current search the system is.
% 
% \subsubsection{The Tolerance Principle}
% To meet the tolerance principle the user interface has been designed to ensure that any user input is either restricted to valid inputs or validated on acceptance, notifying the user of any invalid input.
% A good example of this is the way the user selects the Search Engine, Suitability Measure and Search Problem.
% These selections use drop-down lists.
% This ensures that, assuming a correct implementation, any selection made by the user is valid.
% 
% For user input that is not limited to a small finite set of inputs, all user inputs are checked on and any incorrect input is brought to the users attention.
% This tolerance ensures that the system allows the user to correct any problems without loosing any other input.
% 
% As is noted in Section \ref{sec:introtoquantcomp} all quantum gates must be unitary.
% To ensure the system complies with this, when creating custom matricies for custom gates, the editor checks the matrix to ensure that it represents a unitary operation.
% 
% 
% 
% \subsubsection{The Reuse Principle}
% 
% Visualisation of similar information is shown using the same techniques.
% The use of tabs is two-fold.
% In the column chart panels it is used to select the data to show using the chart.
% In the lowest panel it is used to switch between the different representations of the solution that the system provides.
% 
% 
% \begin{itemize}
%  \item Extendible Library
% \end{itemize}