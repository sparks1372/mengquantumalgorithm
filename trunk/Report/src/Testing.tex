\chapter{Testing}
\label{sec:testing}
\section{Unit Tests}
\label{sec:unittests}
Many of the components created for the framework were tested by unit testing.
With the framework created in Java, JUnit 4\cite{junitweb} was used to produce and run the unit tests.
Not all components were suitable for unit testing due to them being less functional and more control classes.
Therefore unit tests were not created for these components but they were tested as part of the integration testing.

The components that were tested are listed below:
\begin{itemize}
 \item Complex class
 \item Matrix class
 \item All gate implementations
 \item Circuit implementation
 \item Expnode implementation
 \item Algorithm implementation
 \item Suitability measures
 \item Manager classes
\end{itemize}

In the unit tests the test values were chosen in accordance with partition testing to ensure that the test were more complete and tested for the incorrect and the correct partitions with equal weight.

\subsection{Complex Number Implementation Unit Tests}
Due to the Complex class underpinning the functionality of the framework a thorough test was required; reliance on the testing performed on the third party library was to much of a risk.
The unit tests were simple with a range of positive, negative, integer and decimal values used in the construction of the complex classes.

The test oracles used for the \emph{toString} and \emph{parseComplex} methods were created in combination with the other of these two methods.
Additional test oracles were also used but this mutal dependance ensures that the strings produce by the \emph{toString} method are well formed and can be parsed by \emph{parseComplex}.
It also ensures that the \emph{parseComplex} method is correct and creates a Complex object that represents the string passed as its argument.

For the numeric methods, Octave\cite{octweb} was used to create test oracles.

\subsection{Matrix Implementation Unit Tests}
The Matrix is based on a third party implementation but due to both its importance to the correct operation of the framework and the extent of the modification made to it, extensive unit testing was performed.
Even though the framework only requires the correct operation of unitary matrices the testing was not restricted to unitary matrices.
Test matrices were created as real only,  imaginary only and real and imaginary complex values with both positive and negative values.

The Matrix unit tests include testing of the MatrixUtil class that provides the encoding and decoding of matrices to XML files and tensor product of two matrices.

For the arithmetic methods, Octave\cite{octweb} was used to create test oracles.

\subsection{Gate Implementation Unit Tests}
\label{sec:gatetests}
Unit testing the gate implementations is a relatively simple process.
The correct operation of the all gates, excluding the custom gates, are statically defined as in Table \ref{tab:providedgates}.
Therefore creating the unit tests are a series of tests to ensure that with predefined input states, the output states are correct with respect to the gate's definition.
To reduce the probability of errors being introduced into the test oracles used by the unit tests, the test oracles were calculated by Octave\cite{octweb} using Quantum Computing Functions(QCF) for Matlab\cite{qcfweb}.

QCF does not provide all of the gates defined in Table \ref{tab:providedgates}.
For the missing gates, Matlab functions were created using the definitions in Table \ref{tab:providedgates}.

To test the custom gate implementation XML files were created with the definitions of all single qubit gates.
Custom gates were created using these XML files and all the unit tests for the respective gate were applied.
This ensured that the operation of the custom gate is functionally equivalent to the non-custom gates.
The custom gate is not going to ne non-functionally equivalent as some of the non-custom gates exploit bit manipulation rather than matrix multiplication, see Section \ref{sec:custgates}, where as the custom gate can only perform matrix multiplication.

Alongside the \emph{apply} operation performed on quantum states, all other methods are also tested by the unit tests.
The test oracles are much simpler to define.
The \emph{getTarget} method is required to return the integer value set as the target qubit ID in the constructor.
A number of test cases including positive, negative and zero values.
Only positive values should be accepted by the constructor as there are no qubits with a negative or zero ID.
The returned value of each of the other operations are known either implicitly or explicitly by Table \ref{tab:providedgates} and the QCircuit\cite{QCsite} documentation.

\subsection{Circuit Implementation Unit Tests}
\label{sec:circtests}
The unit tests for the circuit implementation followed the following scenarios:
\begin{itemize}
 \item Add a gate
 \item Add a subcircuit
 \item Add a gate and ensure the Latex representation is correct
 \item Add a subcircuit and ensure the Latex representation is correct
 \item Add a gate and ensure the circuit size is correct
 \item Add a subcircuit and ensure the circuit size is correct
 \item Add a gate and ensure the gate in the iterator is correct
 \item Add a subcircuit and ensure the gates in the iterator are correct and in the correct order
\end{itemize}

\subsection{Expnode Implementation Unit Tests}
These unit tests concern the implementation of the Expnode context free grammar shown in Table \ref{tab:expnodecontext}.
The unit tests were based on the expressions that can be seen in Figure \ref{tab:expnodetestexps}.
A collection of values are used as SystemSize including positive, negative, integer and decimal values.
LoopVars shal also be set as one of a collection of arrays with lengths $0$, $1$, $2$ and $3$.
The $0$ length array is important as there is no restriction placed on the search engines stating that the loop variables can only be requested in a loop.
This is to that the search is not restricted.
If the array is of length $0$ the result is $0$ irrespective of the index requested.
If the index requested is greater than the length of the array, modulus is used with the array length to produce a valid index.
This is defined in Section \ref{sec:quantalgs}.

The test oracles are calculated in the test cases to ensure that precision rounding is handled by the test cases.

\begin{figure}
\centering
 \begin{tabular}{|c|c|c|c|}
\hline
$0$ & $2+SystemSize$ &  $LoopVars[0]$ &$2+LoopVars[2+SystemSize]$\\
$2$ & $2-SystemSize$ &   $LoopVars[1]$ & $2-LoopVars[2+SystemSize]$\\
$-2$ & $2*SystemSize$ &   $LoopVars[-1]$ & $2*LoopVars[2+SystemSize]$ \\
&$2/SystemSize$ &  $LoopVars[SystemSize]$ & $2/LoopVars[2+SystemSize]$ \\
&&  $LoopVars[2+SystemSize]$ & \\
\hline
 \end{tabular}
\caption{Expnode Test Expressions}
\label{tab:expnodetestexps}
\end{figure}

\subsection{Algorithm Implementation Unit Tests}
\label{sec:algtests}
The unit tests for the algorithm implementation followed the following scenarios:
\begin{itemize}
 \item Add an instruction
 \item Add four different instructions
 \item Add a instruction and ensure the algorithm size is correct
 \item Add four different instructions and ensure the algorithm size is correct
 \item Add a instruction and ensure the instruction in the iterator is correct
 \item Add four different instructions and ensure the instructions in the iterator are correct and in the correct order
 \item Add a instruction and ensure the printed algorithm is correct
 \item Add four different instructions and ensure the printed algorithm contains the correct instructions, including correct expressions to calculate the numeric vales, are correct and they are printed in the correct order
\end{itemize}

\subsection{Suitability Measure Unit Tests}
\label{sec:suitmeastests}
There are three suitability measures provided with the framework, see Section \ref{sec:provsuitmeas}.
Each of these needed to be tested to ensure that the implementation was consistent with their definitions.

A series of $2^n\times1$ matrices were created and provided to the suitability measure.
The theoretical value, the test oracle, was calculated using Oracle.
The matrices used in the testing were not necessarily correct quantum states, they may not have a modulus square equal to $1$.
This is because even though the operation of gates are by definition unitary, the starting state nor the expected final states defined in test cases are restricted to quantum states that have a modulus square equal to $1$.
As a result the suitability measure tests were not restricted to ``correct'' quantum states.

\subsection{Test Suite Unit Tests}
\label{sec:testsuitetests}
The test suite data structure is a combination of three classes not covered by other unit tests.
Although the test will cover three classes this is still being classed as a unit test as it is effectively a test of the data structure unit rather than the classes.

The tests covered the following scenarios:
\begin{itemize}
 \item (Three tests) Create a new test suite and insert a single test case for 1/2/3 qubits, check test suite data structure against the oracle.
 \item Create a new test suite and insert a single test case for 1 qubit and another test case for 2 qubits, check test suite data structure against the oracle.
 \item Create two new test suite (A and B), insert a single test case for 1 qubit in test suite A and 1 qubit in test suite B, add the test set from test suite B to test suite A. Check test suite A and B data structures against the oracles.
 \item Create a new test suite and insert a single test case for 1 qubit and another test case for 2 qubits, encode the test suite as an XML file, check the test suite XML file against the oracle.
 \item Decode a predefined test suite XML file, check the test suite data structure against the oracle.
\end{itemize}

The scenario consisting of two test suites checks both test suites to ensure that the merge of test sets, and therefore the modification of labels and IDs, does not effect the test suite that is not being modified.

\subsection{Manager Classes Unit Tests}
\label{sec:manclasstests}
For each of the manager classes, a series of XML configuration files were created for testing purposes only.
For the Search Engine and Suitability Measure Manager classes the tests contained several checks:
\begin{itemize}
 \item Check the list of available implementations against the oracle
 \item Select each implementation in turn and check against the oracle the class of the object provided by the manager
\end{itemize}

For the Problem Manager the checks were slightly more in depth due to the returned object containing a test suite data structure.
The test suite implementation includes an \emph{equal} method which provides a ``deep equality'' check.
This is used by the unit test to ensure that the object created by the manager against an oracle.

\section{Integration Tests}
As mentioned in Section \ref{sec:unittests}, the framework contains classes that coordinate the interaction between the functional classes.
The integration tests were designed to test these classes in particular.
These classes in particular were:
\begin{enumerate}
 \item The circuit builder implementation - \emph{basiccircuitbuilder}
 \item The circuit evaluator implementation - \emph{basiccircuitevaluator}
\end{enumerate}

The approach that was taken was a bottom-up approach.
The tests were also performed in the order they are listed.
This is becuase the circuit builder is used by the circuit evaluator and therefore it imposes a dependency.

\subsection{Circuit Builder Integration Tests}
The circuit builder integration test combined:
\begin{itemize}
 \item Gate implementations
 \item Circuit implementation
 \item Algorithm implementation
\end{itemize}

Due to the simplicity of the circuit builder interface, the tests that were carried out were also simple.
The circuit builder interface provides two methods.
Both methods are intended to perform the same action, to take a quantum algorithm and to return the circuit for the specified system size.
The difference between the two methods is that one allows an integer array to be passed as an argument.
This integer array is the \emph{LoopVars} array used in the Expnode grammar, see Section \ref{sec:quantalgs}.
When the method without this argument initialises the \emph{LoopVars} array to the empty array.

To test the two methods, a collection of simple quantum algorithms were produced to include each gate instruction at least one and each of the iterate control instructions at least five times.
These algorithms are passed to the circuit builder, the returned circuits are checked against the test oracles.
The test oracles are circuits that represent the circuit that would be created by the algorithm.

The tests are run over system sizes of 1, 2 and 3.
Using a code review, the method with the additional integer array parameter will be tested by the other method when the algorithm includes iterate control instructions.
This is the reason which each gate was included at a minimum of once but the iterate control instruction were included at least five times.

\subsection{Circuit Evaluator Integration Tests}
\label{sec:cirevaltests}
The circuit evaluator integration test combined:
\begin{itemize}
 \item Circuit builder implementation
 \item Suitability measure implementation
 \item Test suite implementation
\end{itemize}

The suitability measure that was used for the tests was the Simple Suitability Measure, see Section \ref{sec:simplsuitmeas}.

The tests for the circuit evaluator were very similar to those used to test the individual suitability measures, see Section \ref{sec:suitmeastests}.
A series of algorithms were produced alongside a collection of small test suites, containing less that three test cases.
The theoretical suitability value of each algorithm was produced and then compared with the value produced by the circuit evaluator.

Two additional tests were produced to test the \emph{getResults} and \emph{getTrace} methods.
The test oracles were test suites and arrays of test suites respectively.
To ensure that the tests were simple enough that confidence was improved in the test oracle structures that had to be produced manually.
The quantum states in the test oracles that represented the result of correct complete or partial circuit evaluation were produced using Octave also to improve test confidence.

\section{System Testing}
\label{sec:systests}

\section{Client GUI Testing}
\label{sec:clientguitests}
The client GUI that is provided alongside the framework was tested using scenario based testing.
For each of the requirements for the client GUI, see Section \ref{sec:clientguireqs}, had a scenario created specifically for the requirement.

Analysing all the test cases produced, many could be combined producing a much smaller number of test cases without reducing the requirement coverage.
The test cases were analysed for a second time to identify those that could be automated.
The tool chosen for the automated was WindowTester Pro\cite{wintestproweb}.
However, it was decided that the amount of automation that was possible and the limitations of what could be checked during the automated tests did not make it a sensible to proceed with automation of the GUI testing.

\clearpage
\begin{landscape}
\section{Tracability}
\label{sec:tracability}

\centering
\noindent\begin{longtable}{|c|m{10cm}|c|m{3cm}|m{3cm}|}
\hline
\textbf{Requirement ID} &
\multicolumn{1}{c|}{\textbf{Requirement Title}} &
\textbf{Full Requirement} &
\textbf{Addressed by Design} &
\multicolumn{1}{c|}{\textbf{Addressed by Test}} \\ \hline

Req:ASE &
The framework shall allow researchers to provide search engines for the system to use. &
Page \pageref{sec:reqase} &
Section \ref{sec:mulsearchen} &
\multicolumn{1}{c|}{Section \ref{sec:manclasstests}}\\ \hline

Req:ASM &
The framework shall allow researchers to provide suitability measures for the system to use. &
Page \pageref{sec:reqasm} &
Section \ref{sec:mulsuitmeas} &
\multicolumn{1}{c|}{Section \ref{sec:manclasstests}}\\ \hline

Req:QAO &
The solution of a search, a quantum algorithm, shall be presented to the user as a list of instructions. &
Page \pageref{sec:reqqao} &
Section \ref{sec:quantalgs} &
\multicolumn{1}{c|}{Section \ref{sec:algtests}} \\ \hline

Req:CV &
The system shall provide visualisation of the circuit produced by the solution of the search for a system of a user specified number of qubits. &
Page \pageref{sec:reqcv} &
Section \ref{sec:quantumcircuits} &
\multicolumn{1}{c|}{Section \ref{sec:circtests}} \\ \hline

Req:TPS &
The framework shall be able to be embedded in third party software. &
Page \pageref{sec:reqtps} &
Section \ref{sec:thirdpartyblackbox} &
Code Review of Client to ensure only interface knowledge is required\\ \hline

Req:DST &
The framework shall provide a standardised definition format for users to specify the target of the search. &
Page \pageref{sec:reqdst} &
Section \ref{sec:testsuitestruc} &
\multicolumn{1}{c|}{Section \ref{sec:testsuitetests}} \\ \hline

Req:UCF &
The customisation of the framework shall be provided through a series of configuration files. &
Page \pageref{sec:requcf} &
Section \ref{sec:manclasses} &
\multicolumn{1}{c|}{Section \ref{sec:manclasstests}} \\ \hline

Req:PGAI &
The framework shall provide implementations of all gates specified in Table \ref{tab:providedgates}. The framework shall provide algorithm instructions for each of these gates and for the instantiation of the Controlled-U gate with all single qubit gates. &
Page \pageref{sec:reqpgai} &
Sections \ref{sec:quantalgs}, \ref{sec:quantumgates} &
\multicolumn{1}{c|}{Sections \ref{sec:gatetests}, \ref{sec:algtests}} \\ \hline

Req:ACS &
The system shall provide the iterate control structure and support nested iterate instructions. &
Page \pageref{sec:reqacs} &
Section \ref{sec:quantalgs} &
\multicolumn{1}{c|}{Sections \ref{sec:gatetests}, \ref{sec:algtests}} \\ \hline

Req:PCA &
The framework shall be able to produce a circuit, for any given number of qubits, from a quantum algorithm. &
Page \pageref{sec:reqpca} &
Sections \ref{sec:quantumcircuits}, \ref{sec:quantumgates} &
Code review of both the circuit implementation and all of the gate implementations to ensure the logic does not rely on upper bound to the system size.
The only restriction that is present is where $2^{System Size}>Integer.MAX\_VALUE$. \\ \hline

Req:CS &
The framework shall provide the simulation of a circuit given an initial state. &
Page \pageref{sec:reqcs} &
Section \ref{sec:quantumgates} &
\multicolumn{1}{c|}{Sections \ref{sec:gatetests}, \ref{sec:cirevaltests}} \\ \hline

Req:SBSSE &
The framework shall provide a way to perform step-by-step evaluation of a circuit given an initial state. &
Page \pageref{sec:reqsbsse} &
Section \ref{sec:sbsecireval} &
\multicolumn{1}{c|}{Section \ref{sec:cirevaltests}} \\ \hline

&&&&
\\ \hline

Req:SSE &
The tool shall provide at least one implemented search engine. &
Page \pageref{sec:reqsse} &
Section \ref{sec:provsearcheng} &
\multicolumn{1}{c|}{Section \ref{sec:manclasstests}} \\ \hline

Req:SSM &
The tool shall provide at least one implemented suitability measure. &
Page \pageref{sec:reqssm} &
Section \ref{sec:provsuitmeas} &
\multicolumn{1}{c|}{Section \ref{sec:manclasstests}} \\ \hline

Req:SST &
The tool shall provide a number of search targets with known outputs. &
Page \pageref{sec:reqsst} &
Section \ref{sec:provsearchprobs} &
\multicolumn{1}{c|}{Section \ref{sec:manclasstests}} \\ \hline

Req:SES &
The GUI shall provide a user with a selection of search engines to use in a search. &
Page \pageref{sec:reqses} &
Section \ref{sec:seselecdes} &
\multicolumn{1}{c|}{Section \ref{sec:clientguitests}} \\ \hline

Req:SMS &
The GUI shall provide a user with a selection of suitability measures to use in a search. &
Page \pageref{sec:reqsms} &
Section \ref{sec:smselecdes} &
\multicolumn{1}{c|}{Section \ref{sec:clientguitests}} \\ \hline

Req:STS &
The GUI shall provide a user with a selection of search targets to be used as the search goal. &
Page \pageref{sec:reqsts} &
Section \ref{sec:spselecdes} &
\multicolumn{1}{c|}{Section \ref{sec:clientguitests}} \\ \hline

Req:STC &
The GUI shall provide a way for users to create a new search target without needing to explicitly write a configuration file. &
Page \pageref{sec:reqstc} &
Section \ref{sec:guisearchcreateed} &
\multicolumn{1}{c|}{Section \ref{sec:clientguitests}} \\ \hline

Req:STE &
The GUI shall provide a way for users to edit the contents of a previously created search target without manual editing of the configuration file. &
Page \pageref{sec:reqste} &
Sections \ref{sec:indtestsuiteeditor}, \ref{sec:guisearchcreateed} &
\multicolumn{1}{c|}{Section \ref{sec:clientguitests}} \\ \hline

Req:LSTPDC &
The GUI shall provide a way to import a predefined search target from a configuration file. &
Page \pageref{sec:reqlstpdc} &
Section \ref{sec:guisearchcreateed} &
\multicolumn{1}{c|}{Section \ref{sec:clientguitests}} \\ \hline

Req:SV &
The GUI shall provide a way to visualise any quantum state. &
Page \pageref{sec:reqsv} &
Section \ref{sec:repres} &
\multicolumn{1}{c|}{Section \ref{sec:clientguitests}} \\ \hline

Req:RSR &
The GUI shall provide a way to report the search result, a quantum algorithm, to the user. &
Page \pageref{sec:reqrsr} &
Section \ref{sec:repres} &
\multicolumn{1}{c|}{Section \ref{sec:clientguitests}} \\ \hline

Req:GCV &
Given a quantum algorithm and a system size, the GUI shall produce a visualisation of the resulting circuit. &
Page \pageref{sec:reqgcv} &
Section \ref{sec:repres} &
\multicolumn{1}{c|}{Section \ref{sec:clientguitests}} \\ \hline

Req:GSBSSE &
The GUI shall provide a way to perform, control and visualise the step-by-step state evolution for an intial state and circuit. &
Page \pageref{sec:reqgsbsse} &
Section \ref{sec:repres} &
\multicolumn{1}{c|}{Section \ref{sec:clientguitests}} \\ \hline

Req:TT &
The GUI shall provide user help through the use of tooltips. &
Page \pageref{sec:reqtt} &
Implemented for all buttons&
\multicolumn{1}{c|}{Witnessed during Section \ref{sec:clientguitests}} \\ \hline

Req:POR &
The framework, fully implemented tool and the GUI shall be able to be used on a range of Operating Systems. &
Page \pageref{sec:reqpor} &
Property of Java, Section \ref{sec:reqs} &
\multicolumn{1}{c|}{Section \ref{sec:clientguitests}} \\ \hline

\end{longtable}
\end{landscape}
\clearpage